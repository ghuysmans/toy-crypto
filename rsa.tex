\documentclass[11pt,twocolumn]{article}
\usepackage{fullpage}
\usepackage[a4paper, margin=1.5cm, bottom=3cm]{geometry}
\usepackage{fourier} %fonte plus lisible
\usepackage{hyperref}
\usepackage[T1]{fontenc}
\usepackage[francais]{babel}
\usepackage[utf8]{inputenc}
\usepackage{amsmath}
\let\mod\undefined
\DeclareMathOperator{\mod}{mod}
\usepackage{amsthm}
\theoremstyle{plain}
\newtheorem{df}{Définition}
\newtheorem{pr}{Propriété}
\newtheorem{thm}{Théorème}
\newcommand{\esP}{\mathbb{P}} %premiers
\newcommand{\esN}{\mathbb{N}} %naturels
\newcommand{\esZ}{\mathbb{Z}} %entiers
\newcommand{\esF}{\mathbb{F}} %corps fini
\newcommand{\dbi}{\Longleftrightarrow}
\usepackage[fixlanguage]{babelbib}
\selectbiblanguage{french}
\usepackage{cite}
\usepackage{enumerate}
\newenvironment{cproof}[1]{\begin{proof}[Démonstration \cite{#1}]}{\end{proof}}

\title{Cryptographie asymétrique}
\author{Guillaume \textsc{Huysmans}}
\hypersetup{pdfauthor={Guillaume Huysmans},
	pdftitle={Cryptographie asymétrique},
	pdfsubject={algèbre modulaire, sécurité, cryptographie moderne},
	pdfkeywords={cryptographie, algèbre modulaire, sécurité, maths, math}}

\begin{document}
\maketitle

Ce document synthétise
\emph{de manière incomplète et sûrement truffée d'erreurs}
des maths abordées en MAB1 en Sciences Informatiques à l'UMONS.
N'hésitez pas à me transmettre vos remarques,
commentaires \& suggestions par e-mail !

La dernière version de ce document et de l'outil-jouet
\footnote{J'en déconseille toute utilisation sérieuse
	(\emph{timing attack}, stockage simpliste,
	non-vérification des clés...)}
qui l'accompagne est disponible en ligne : \\
\url{https://github.com/ghuysmans/toy-crypto}


\section{Théorie des nombres}
On se permettra d'écrire
$[5;42[ = \left\{n \in \esN \; \middle| \; 5 \leq n < 42\right\}$ :
de toute façon, les réels ne servent à rien ici.

\begin{df}[Divise] Soient $a, b \in \esZ$. \[
	a/b \dbi \exists k \in \esZ : b = ka
\] \end{df}
\begin{df}[PGCD] Soient $a, b \in \esZ$. \[
	(a,b) = \min \left\{n \in \esN \; \middle| \; n/a \land n/b\right\}
\] \end{df}
\begin{pr}\label{pr:min}
	Soient $a, b \in \esZ$.
	Soit $I = \left\{ax + by \; \middle| \; x,y \in \esZ\right\}$. \[
		(a,b) = \min (I \cap \esN_0)
	\]
	\begin{cproof}{Buys}
		Soit $d = \min (I \cap \esN_0)$.
		\begin{enumerate}
			\item $d$ divise $a$ :
				\begin{itemize}
					\item par définition,
						$\exists u, v \in \esZ : au + bv = d$
					\item par l'absurde, supposons $\lnot(d/a)$ : \\
						$\exists r \in \; ]0;d[ \; : a = dq + r$
					\item $r=a-dq=a-(au+bv)q=a(1-uq)+b(-vq)$
					\item par définition, $r \in I$
					\item contradiction (FIXME où ça ?)
				\end{itemize}
			\item $d$ divise $b$ :
				\begin{itemize}
					\item par l'absurde, supposons $\lnot(d/b)$...
					\item $r=b-dq=b-(au+bv)q=b(1-vq)+a(-uq)$
				\end{itemize}
			\item Soit $d' \in \esZ$ tel que $d'/a$ et $d'/b$. $d' \leq d$ : \\
				$d'$ divise chaque terme de $au+bv=d$
		\end{enumerate}
		Par définition, $d=(a,b)$.
	\end{cproof}
\end{pr}
\begin{pr} Soient $a, b \in \esZ, m \in \esN_0$. \[
	(a, m) = 1 \land (b, m) = 1 \implies (ab, m) = 1
\] \end{pr}
\begin{thm}[Euclide]
	Soient $a \geq b \in \esZ$. \[
		(a, b) = \left\{
			\begin{array}{ll}
				a & \text{si } b=0 \\
				(b, a \; \mod \; b) & \text{sinon} \\
			\end{array}
		\right.
	\]
	\begin{cproof}{Conlen} ~
		\begin{itemize}
			\item[$b=0$] Trivial.
			\item[$b\neq0$]
				$\exists q, r \in \esZ : a = qb+r$ (division avec reste). \\
				Soient $x=(a,b), y=(b,r)$.
				\begin{itemize}
					\item $x/y$ : $x/a$ et $x/b$, donc $x/qb$ et $x/(a-qb=r)$.
					\item $y/x$ : $y/b$ et $y/(a-qb)$, donc $y/a$.
				\end{itemize}
				On en déduit que $x=y$.
		\end{itemize}
		On peut ainsi itérativement calculer un PGCD en recommençant avec
		$(b,r)$ jusqu'à ce que $b=0$.
	\end{cproof}
\end{thm}
\begin{thm}[Bézout]
	Soient $a, b \in \esZ$ ($a=0 \oplus b=0$). \[
		\exists u, v \in \esZ : au+bv = (a, b)
	\]
	\begin{proof}[Bah oui !]
		Par la propriété \ref{pr:min}, ils existent puisque le PGCD est dans $I$
		et que ses éléments s'écrivent sous cette forme.
		\emph{(Pas très constructif, tout ça...)}
	\end{proof}
	\begin{proof}[Euclide étendu \cite{Gillis}]
		Soient $a \geq b \geq 0$ :
		\begin{itemize}
			\item Cas de base $b=0$. Prenons $u=1, v=0, (a,0)=a$.
			\item Pas de récurrence : $a=bq+r$ par division euclidienne.
				Supposons (hypothèse de récurrence) que
				ça fonctionne pour $\forall b'<b$.
				Utilisons-la avec $a'=b, b'=a\mod b<b$, elle nous donne : \[
					\exists u', v' \quad bu'+rv' = (b,r)
				\]

				Prenons $u=v', v=u'-qv'$.
				Par le théorème d'Euclide, $(a,b)=(b,r)$.
				Montrons que :
				\begin{align*}
					au+bv&=(a,b) && \text{thèse} \\
					av'+b\left(u'-qv'\right)&=(b,r) && \text{substitutions} \\
					v'\left(a-bq\right)+bu'&=(b,r) && \text{mise en évidence} \\
					v'r+bu'&=(b,r) && r=a-bq
				\end{align*}

				C'est notre hypothèse de récurrence.
		\end{itemize}
	\end{proof}
\end{thm}
\newpage
\begin{pr}[Inverse modulaire]
	Soient $a \in \esZ, m \in \esN_0$. \[
		\exists x \in \esZ : ax \equiv_m 1 \dbi (a,m)=1
	\]
	\begin{proof} ~
		\begin{itemize}
			\item[$\Rightarrow$]
				Soit $x \in \esZ$.
				Supposons $\exists y \in \esZ : ax-ym=1$.
				Par la propriété \ref{pr:min},
				$(a,m)=1$ puisque $\min(\esN_0)=1\in I$.
			\item[$\Leftarrow$]
				Par Bézout (puisque $m \neq 0$), on sait que
				$\exists u, v \in \esZ : au+mv = (a, b) = 1$.
				Prenons $x=u$. On a bien $\exists y \in \esZ : au-ym=1$,
				il suffit de prendre $y=-v$.
		\end{itemize}
	\end{proof}
	\begin{proof}[Une autre \cite{Buys}] ~
		\begin{itemize}
			\item[$\Rightarrow$]
				Par définition, $d/a$ et $d/m$.
				Par hypothèse (réécrite), $m/(ax-1)$.
				Par transitivité, $d/(ax-1)$.
				Forcément, $d/ax$.
				Donc $d/1$, c'est-à-dire $d=1$.
			\item[$\Leftarrow$]
				Par Bézout, $\exists u, v \in \esZ : au+mv \equiv_m 1$.
				Les $mv$ se font tuer par le modulo, il reste
				$\exists u \in \esZ : au \equiv_m 1$.
		\end{itemize}
	\end{proof}
\end{pr}
\begin{pr}[Inverse unique]
	Soient $a, x, x' \in \esZ, m \in \esN_0$. \[
		(ax \equiv_m 1) \land (ax' \equiv_m 1) \implies x \equiv_m x'
	\]
	\begin{cproof}{Buys}
		$a(x-x') \equiv_m 1$.
		Une règle analogue à celle du produit nul s'applique :
		soit $a \equiv_m 0$ (impossible car $(a,m)=1$),
		soit $x-x' \equiv_m 0$.
	\end{cproof}
\end{pr}
\begin{thm}[Fermat]\label{thm:fermat}
	Soient $a \in \esN, p \in \esP$. \[
		a^p \equiv_p a
	\]

	Sans perte de généralité, supposons que $(a,p)=1$
	(sinon, ça reste vrai : $0 \equiv_p 0$).
	\begin{proof}[Transformation \cite{Gillis}]
		\begin{align*}
			a^pa^{-1} & \equiv_p aa^{-1} && \text{multiplication} \\
			a^{p-1}1 & \equiv_p 1 && aa^{-1} \equiv_p 1 \\
			a^{p-1} & \equiv_p 1 && 1 \text{ neutre}
		\end{align*}
	\end{proof}
	\begin{proof}[Permutations \cite{Buys}]
		Construisons un tableau en partant d'une ligne avec tous les éléments
		du corps fini $\esF_p^* = \esF_p \setminus \{0\}$ et regardons une
		ligne des $a^{\text{èmes}}$ multiples de la première :
		$\left[a, 2a, 3a \dots \left(p-1\right)a\right] \mod p$.

		Aucun élément de la première ligne n'est multiple de $p$ (on s'arrête à
		$p-1$) et le produit de deux non multiples (ces éléments et le
		multiplicateur de chaque ligne) reste non multiple : aucun élément ne
		sera nul. Il n'y a pas non plus de répétitions :
		\begin{align*}
			\exists x \neq y \qquad\qquad
			xa & \equiv_p ya && \text{par l'absurde} \\
			xa - ya & \equiv_p 0 && \text{changement de côté} \\
			a \left(x - y\right) & \equiv_p 0 && \text{mise en évidence} \\
			x & \equiv_p y && \lnot \left(a \equiv_p 0\right)
		\end{align*}

		Le produit des éléments d'une ligne est constant (chacune est donc une
		permutation de la première) :
		\begin{align*}
			(1a)(2a)\dots\left(p-1\right)a
			& \equiv_p \left(p-1\right)!a^{p-1} && \text{regroupements} \\
			& \equiv_p \left(p-1\right)! && a=1 \implies 1^{p-1} = 1
		\end{align*}

		On peut simplifier les factorielles puisque
		leurs facteurs inférieurs à $p$ ne seront pas <<~nuls~>>.
	\end{proof}
	\begin{proof}[Binôme de Newton \cite{Buys}]
		Dans le triangle de Pascal, pour $p \in \esP$ (ligne) et
		$k \in ]0;p[$ (colonne), on ne trouve que des multiples de $p$
		à l'intérieur. \emph{Joli, hein ?}

		\[
			\binom p k = \frac{p!}{\left(p-k\right)!k!}
		\]

		\begin{align*}
			a^p
			&= \left(1+\left(a-1\right)\right)^p && \text{artifice de calcul} \\
			&= \sum_{k=0}^p\binom pk(a-1)^k1^{p-k} && \text{binôme de Newton} \\
			&\equiv_p 1 + (a-1)^p && \text{TODO pourquoi ?}
		\end{align*}

		On calcule ça $\forall i \in [1;a]$ :
		\begin{align*}
			1^p &\equiv_p 1+0 \\
			2^p &\equiv_p 1+1^p \\
			3^p &\equiv_p 1+2^p \\
			& ~~\vdots \\
			(a-1)^p &\equiv_p 1+(a-2)^p \\
			a^p &\equiv_p 1+(a-1)^p
		\end{align*}

		On somme tout :
		\begin{align*}
			\left(1^p + 2^p \cdots + (a-1)^p\right) + a^p
				&\equiv_p a + \left(1^p + 2^p \cdots + (a-1)^p\right) \\
			a^p &\equiv_p a
		\end{align*}
	\end{proof}
\end{thm}
\begin{pr}[Test de Fermat]
	Soit $n \in \esN_0$.
	Supposons que $\exists a \in \esN_0 : a^{n-1} \not\equiv_n 1$.
	Par Fermat (contraposée), $n \not\in \esP$.

	Au moins la moitié des entiers plus petits que $p$ ne passeront pas le test.
	En prenant un entier au hasard plus petit que $p$, on a donc une chance
	sur deux de prouver que ce nombre n'est pas premier.

	Les (rares mais infiniment nombreux) nombres de Carmichael
	passeront toujours ce test sans être premiers.
	On peut les tester (TODO p38 ?)
	\begin{cproof}{Gillis}
		\[
			\forall b \in \esN_0 : b^{p-1}\equiv_p1
			\implies
			(ab)^{p-1} \equiv_p a^{p-1} \not\equiv_p 1
		\]
		À tout $b$ qui passe le test, on peut associer $b'=ab$
		qui ne le passe forcément pas.
	\end{cproof}
\end{pr}
\begin{thm}[Lagrange]
	Soit $\pi(x)=\left|\left\{p\in\esP \; \middle| \; p\leq x \right\}\right|$.
	\[
		\lim_{x\to+\infty}\frac{\pi(x)}{x/\ln(x)}=1
		\implies
		\pi(x) \approx \frac x{\ln(x)}
	\]
\end{thm}
\begin{pr}[Génération de premiers]
	\[
		P\left(x \in \esP, x \leq e^n \right) =
		\frac{\pi(x)}x=\frac1{\ln(x)}\approx\frac1n
	\]

	\cite{Gillis} : En moyenne, on devra donc générer
	de l'ordre de $n$ nombres
	avant de trouver un premier à $n$ chiffres.
\end{pr}
\begin{df}[Groupe] $(G, \bullet)$ est groupe si :
	\begin{itemize}
		\item $\bullet$ est une opération binaire...
			\begin{itemize}
				\item interne : \\
					$\forall a, b \in G, \exists c \in G : a \bullet b = c$
				\item associative : \\
					$\forall a, b, c \in G,
					(a \bullet b) \bullet c = a \bullet (b \bullet c)$
			\end{itemize}
		\item $G$ possède un unique élément neutre : \\
			$\exists e \in G, \forall x \in G : x \bullet e = e \bullet x = x$
		\item Tout élément admet un symétrique : \\
			$\forall x \in G, \exists x' \in G : x \bullet x' = e$
	\end{itemize}
\end{df}
\begin{df}[Indicateur d'Euler] Soit $m \in \esN$. \[
	\phi(m) = \left|\left\{n \in \esN \; \middle| \;
	n < m \land (n,m)=1 \right\}\right|
\] \end{df}
\begin{pr} Soit $p \in \esP$. Puisque $(0,p)=p$, \[
		\phi(p)=p-1
\] \end{pr}
\begin{pr}
	Soit $m \in \esN$. \[
		\phi(m) = m \left.\prod_{p/m} \left(1 - \frac1p\right)\right.
		\quad (p \in \esP)
	\]
	\begin{cproof}{Buys} ~
		\begin{itemize}
			\item Les éléments qui ont un facteur commun avec $p^\alpha$ sont
				$p, 2p, 3p \dots p^{\alpha-1}p$, au nombre de $p^{\alpha-1}$ :
				\begin{align*}
					\forall \alpha \geq 2, p \in \esP \quad
					\phi\left(p^\alpha\right)
					&= p^\alpha - p^{\alpha-1} \\
					&= p^\alpha\left(1-\frac1p\right)
				\end{align*}
			\item Si $(m,n)=1$, comptons les nombres
				inférieurs et premiers avec $mn$ à l'aide d'une matrice : \[
					\left(\begin{array}{llll}
						1 & 2 & \dots & m \\
						m+1 & m+2 & \dots & 2m \\
						\vdots & \vdots & \ddots & \vdots \\
						(n-1)m+1 & (n-1)m+2 & \dots & nm
					\end{array}\right)
				\]
				Regardons une colonne $r \leq m$ de la matrice :
				\begin{enumerate}
					\item $(r,m)=1 \implies \forall i \in [0;n-1] : (im+r,m)=1$
					\item Les éléments sont distincts ($\mod n$) :
						par l'absurde,
						supposons $\exists i \neq j : im+r \equiv_n jm+r$.
						On aurait $n/\left(i-j\right)m$, ce qui est impossible
						puisque $(m,n)=1$ et $\left|i-j\right|<n$.
						La colonne est donc une permutation de $[0;\phi(n)]$.
				\end{enumerate}
				Donc, $\phi(mn)=\phi(m)\phi(n)$.
		\end{itemize}
		FIXME comment conclure ?
	\end{cproof}
	\begin{cproof}{Buys}
		On ajoute des facteurs $p$ :
		\begin{itemize}
			\item si $p/m$ (déjà utilisé) :
				\begin{align*}
					\forall a \in \esZ, k \in \esN \quad (mp,a)=1
					&\dbi (m,a)=1 \\
					&\dbi (m,a+km)=1
				\end{align*}
				\[
					[1;mp] = \bigcup_{k=0}^{p-1}[km+1;km+m]
				\]
				Dans chaque ensemble de cette partition,
				$\phi(m)$ éléments sont inférieurs et
				premiers avec $mp$ : \[
					\phi(mp)=p\phi(m)
				\]
			\item sinon, $\lnot(p/m)$, on ne l'a pas encore utilisé :
				\begin{align*}
					\forall a \in \esZ \quad (mp,a)=1
					\dbi (m,a)=1 \land& (p,a)=1 \\
					& \lnot(p/a)
				\end{align*}
				\[
					\phi(mp)=p\phi(m)-N
				\]
				$N$ correspond au nombre de multiples de $p$
				inférieurs à $mp$ et premiers avec $m$.
				On peut montrer (TODO comment ?) que $N=\phi(m)$, d'où :
				\begin{align*}
					\phi(mp) &= p\phi(m) - \phi(m) && \text{substitution} \\
					&= p\phi(m)\left(1-\frac1p\right) && p\frac1p=1
				\end{align*}
		\end{itemize}
	\end{cproof}
\end{pr}
\begin{pr}[Exponentiation rapide]\label{pr:fastexp}
	Soient $a,b \in \esN$.
	On peut calculer $a^b$ en $O(\log_2b)$.
	\begin{proof} ~
		\begin{itemize}
			\item Cas de base :
				\begin{itemize}
					\item $b=0 \implies a^b = 1$ (tant pis si $a=0$)
					\item $b=1 \implies a^b = a$
				\end{itemize}
			\item Pas de récurrence :
				\begin{itemize}
					\item $b=2k \implies a^b =
						\left(a^2\right)^\frac{b}2$
					\item $b=2k+1 \implies a^b =
						aa^{b-1}=a\left(a^2\right)^{\frac{b-1}2}$
				\end{itemize}
		\end{itemize}
	\end{proof}
\end{pr}
\begin{pr}[Exponentiation rapide modulo]
	Soient $a,b \in \esN$ et $m \in \esN_0$.
	$a^b \mod m$ est calculable en $O(\log b)$.
	\begin{proof}[Intuition]
		On évite à chaque étape de générer des nombres trop gros puisque
		la seule chose qui nous intéresse à la fin est leur reste.
	\end{proof}
\end{pr}
\begin{thm}[Euler]\label{thm:euler}
	Soient $x \in \esZ, m \in \esN_0$. \[
		(x,m) = 1 \implies x^{\phi(m)} \equiv_m 1
	\]
	\begin{proof}[Cas particulier]
		Si $x \in \esP$, c'est Fermat (\ref{thm:fermat}).
	\end{proof}
	\begin{cproof}{Buys}
		$G=\left\{a\in\esN \; \middle| \; a \leq m \land (a,m)=1\right\}$
		forme un groupe pour la multiplication $\mod m$ :
		\begin{itemize}
			\item Cette multiplication est interne : le produit (dans $\esZ$)
				de deux nombres premiers avec $m$ le reste.
				Soit ce produit est inférieur à $m$,
				soit on prend son modulo ($ab-km$, toujours premier avec $m$).
			\item Cette multiplication est associative.
			\item $1$ est le neutre du groupe :
				\begin{itemize}
					\item $1\in G$ car $(1,m)=1$
					\item 1 est neutre pour le produit dans $\esZ$
				\end{itemize}
			\item Tout élément est inversible.
		\end{itemize}
		Soient $a$, un des $a_i$ et $P=a_1a_2\dots a_{\phi(m)}$ :
		\begin{align*}
			\left(aa_1\right)\left(aa_2\right)\dots\left(aa_{\phi(m)}\right)
			&\equiv_m a^{\phi(m)}P && \text{associativité} \\
			\left(aa_1\right)\left(aa_2\right)\dots\left(aa_{\phi(m)}\right)
			&\equiv_m P && \text{facteurs distincts} \\
			a^{\phi(m)}P &\equiv_m P && \text{transitivité} \\
			a^{\phi(m)} &\equiv_m 1 && PP^{-1}\equiv_m1
		\end{align*}
		On retrouve bien l'énoncé du théorème.
	\end{cproof}
\end{thm}


\section{Rivest-Shamir-Adleman}
\begin{enumerate}
	\item Choisir $p, q \in \esP$ suffisamment grands et pas trop proches
		pour éviter la force brute autour de $\sqrt n$
	\item Calculer $n=pq$
	\item Calculer $\phi=\phi(n)=\phi(p)\phi(q)=
		\left(p-1\right)\left(q-1\right)$
		%ce sera toujours un multiple de 4.
	\item Choisir $e>1$ tel que $\left(e,\phi\right)=1$ :
		\begin{itemize}
			\item Évidemment, $e=1 \implies \forall M : E(M)=M$...
			\item Selon \cite{Buys} : prendre $e \in \esP, e>\phi$.
				TODO pourquoi ne pas prendre simplement $\phi+1$ ?
		\end{itemize}
	\item Calculer $d$ tel que $ed \equiv_\phi 1$;
		il existe car $\left(e,\phi\right)=1$.
	\item Partager le couple $(n,e)$, notre clé publique
	\item Garder précieusement le couple $(n,d)$, notre clé secrète.
		$p$ et $q$ suffiraient mais
		on ne veut pas refaire inutilement les calculs...
\end{enumerate}

Soit un message $M < n$
(sinon, le modulo va le dégommer mais au pire, on le segmente) :
\begin{align*}
	E(M) &= M^e \mod n \\
	D(M) &= M^d \mod n
\end{align*}

En pratique, c'est souvent une clé qui sera chiffrée/déchiffrée avec
RSA vu sa lenteur par rapport aux chiffrements symétriques.

Pour signer, on chiffre le message (typiquement son empreinte)
avec la clé privée, ce qui permet à n'importe qui de le déchiffrer
(les clés sont interchangeables).

\begin{thm}[RSA]
	\[
		\left(M^e \mod n\right)^d \equiv_n M^{ed} \equiv_n M
	\]
	\begin{cproof}{Buys}
		\begin{align*}
			\exists k \in \esN \quad M^{ed}
			&\equiv_n M^{1+k\phi}
				&& ed\equiv_\phi1 \text{, congruence} \\
			&\equiv_n M\left(M^\phi\right)^k
				&& \text{propriété des exposants} \\
			&\equiv_n M \times 1^k
				&& \text{par Euler (\ref{thm:euler}), } M^\phi\equiv_n 1 \\
			&\equiv_n M
				&& \text{1 neutre}
		\end{align*}
	\end{cproof}
	\begin{cproof}{hac} ~
		\[
			ed\equiv_\phi1 \dbi \exists k \in \esN : ed=1+k\phi
		\]
		\begin{align*}
			M^{p-1} &\equiv_p 1
				&& \text{par Fermat (\ref{thm:fermat})} \\
			M^{k\left(p-1\right)\left(q-1\right)} &\equiv_p 1
				&& \text{même puissance} \\
			M^{1+k\left(p-1\right)\left(q-1\right)} &\equiv_p M
				&& \text{produit avec }m
		\end{align*}

		Si $(M,p)=p$, la dernière étape <<~annule~>> tout
		mais la congruence reste vraie. On trouve : \[
			M^{ed} \equiv_p M
		\]

		Le même raisonnement s'applique à $q$ : \[
			M^{ed} \equiv_q M
		\]

		Comme $p \neq q$ et $(p,q)=1$ : \[
			M^{ed} \equiv_n M
		\]
	\end{cproof}
\end{thm}

\subsection{Optimisations}
\begin{pr}
	Soient $a \in \esN, p \in \esP$ tels que $(a,p)=1$. \[
		a^{p-2} \equiv_p a^{p-1}a^{-1} \equiv_p a^{-1}
	\]
\end{pr}
\begin{pr}
	Soient $a,b,c \in \esN, p \in \esP$. \[
		b \equiv_{p-1} c
		\implies
		a^b \equiv_p a^c
	\]
	\begin{proof}
		\begin{align*}
			\exists k \in \esZ \quad b
			&= k\left(p-1\right)+c && \text{définition} \\
			a^c
			&\equiv_p a^{k\left(p-1\right)+c} && \text{thèse} \\
			&\equiv_p \left(a^{p-1}\right)^ka^c && \text{exposants} \\
			&\equiv_p 1^ka^c && \text{par Fermat (\ref{thm:fermat})}
		\end{align*}
	\end{proof}
\end{pr}
\begin{thm}[Restes chinois]\label{thm:chrem}
	Soient $x,a,b \in \esN$ et $p,q \in \esN_0$ tels que $(p,q)=1$. On a : \[
		\left\{\begin{array}{ll}
			x \equiv_p a \\
			x \equiv_q b
		\end{array}\right.
		\implies
		\left\{\begin{array}{l}
			\exists p' : pp'\equiv_q1 \\
			\exists q' : qq'\equiv_p1 \\
			x \equiv_{pq} aqq'+bpp'
		\end{array}\right.
	\]
	\begin{proof}
		$p', q'$ existent puisque par hypothèse, $(p,q)=1$.
		$y=aqq'+bpp'$ satisfait les deux équations :
		\begin{itemize}
			\item $y \equiv_p aqq' + 0 \equiv_p a$
			\item $y \equiv_q 0 + bpp' \equiv_q b$
		\end{itemize}

		Cette solution est unique modulo $pq$ : soit $z$ tel que
		$z \equiv_p a$ et $z \equiv_q b$. Dès lors, $z-y$ est un multiple de $p$
		et de $q$. Encore une fois, $(p,q)=1$ donc $z-y$ doit être un multiple
		de $pq$, d'où $z \equiv_{pq} y$.

		En fait, il existe une bijection
		$\esZ_p \times \esZ_q \mapsto \esZ_{pq}$...
	\end{proof}
\end{thm}

Selon \cite[p. 611]{hac}, le théorème \ref{thm:chrem} peut être utilisé
pour accélérer le déchiffrement et la signature :
\begin{enumerate}
	\item Décomposer $M$ en un système (calculer $a, b$) : \[
		\left\{\begin{array}{ll}
			M \equiv_p a \\
			M \equiv_q b
		\end{array}\right.
	\]
	\item Élever les congruences à la énième puissance : \[
		\left\{\begin{array}{ll}
			M^n \equiv_p a^n \\
			M^n \equiv_q b^n
		\end{array}\right.
	\]
	\item Résoudre ce système (trouver $M^n \mod pq$)
\end{enumerate}

\begin{thm}[Restes chinois généralisé]\label{thm:chremg}
	Soient $n \in \esN_0$,
	$\forall i\neq j \in [1;n] : (m_i,m_j)=1$,
	$\forall i \in [1;n] : a_i \in \esN$.
	On a : \[
		\left\{\begin{array}{ll}
			x &\equiv_{m_1} a_1 \\
			x &\equiv_{m_2} a_2 \\
			&\vdots \\
			x &\equiv_{m_n} a_n
		\end{array}\right.
		\dbi
		\left\{\begin{array}{l}
			m=m_1m_2\dots m_i \\
			\exists y \in [0;m] \quad x \equiv_m y
		\end{array}\right.
	\]
	\begin{proof}[Algorithme de Gauss] Selon \cite[p. 68]{hac},
		\begin{align*}
			N_i &= m/m_i \\
			M_i &= N_i^{-1} \mod m_i \\
			x &\equiv_m \sum_{i=1}^k a_iN_iM_i
		\end{align*}

		Chaque congruence est vérifiée (TODO rigueur).
		L'expression analytique de $x$ s'allonge vite...
	\end{proof}
\end{thm}

\subsection{Le danger d'un petit $e$}
Selon \cite[p. 288]{hac}, il est intéressant de prendre un nombre avec
peu de bits à 1 pour accélérer le chiffrement, par exemple 3 ou 65537...
Il faut quand même veiller à respecter la condition : pour 3, ni $p-1$ ni $q-1$
ne peuvent être divisibles par 3.

Si le même message est envoyé à trois destinataires avec le même $e=3$,
il est fort probable que $n_1,n_2,n_3$ soient premiers entre eux.
Posons ce système à partir des messages interceptés : \[\left\{\begin{array}{ll}
	M^3 \equiv_{n_1} a \\
	M^3 \equiv_{n_2} b \\
	M^3 \equiv_{n_3} c
\end{array}\right.\]

En utilisant le théorème \ref{thm:chremg},
on peut retrouver $M^3 \mod n_1n_2n_3$ et en calculer la racine cubique entière.
Le modulo ne gênera pas puisque $M^3<n_1n_2n_3$ !

Pour éviter ce problème, il suffit de \emph{saler} le message envoyé,
c'est-à-dire y insérer des bits pseudo-aléatoires.


\section{Diffie-Hellman}
\begin{enumerate}
	\item Alice et Bob conviennent d'un $p \in \esP$ suffisamment grand
		et d'un générateur $g$ de $\esF_p^*$ (publics).
	\item Alice génère et stocke $a<p$ et envoie $g^a \mod p$.
	\item Bob génère et stocke $b<p$ et envoie $g^b \mod p$.
	\item Chacun calcule de son côté la clé partagée
		$g^{ab} \mod p = r^s \mod p$,
		avec $r$ ce qui a été reçu et $s$ son secret.
		Elle sera identique ($ab=ba$ à l'exposant)
		sans jamais avoir été partagée explicitement.
\end{enumerate}

Attention : Alice pourrait parler à Ève, placée entre elle et Bob, en croyant
parler à Bob. Ève n'aurait qu'à faire l'échange de clé de chaque côté.
Il est nécessaire de signer les messages échangés
afin de pouvoir vérifier de chaque côté d'où ils proviennent.


\section{El Gamal}
\begin{enumerate}
	\item Choisir $p \in \esP$ suffisamment grand
	\item Choisir $\alpha$, un générateur de $\esF_p^*$.
		Pas forcément évident, voir \cite[p. 163--164]{hac}.
	\item Choisir et garder $a \in [1;p-2]$, notre clé secrète
	\item Calculer $\beta=\alpha^a \mod p$
	\item Partager notre clé publique $p, \alpha, \beta$
\end{enumerate}

Chiffrement : choisir $k \in [1;p-2]$ puis calculer
\begin{align*}
	M_1 &= \alpha^k \mod p \\
	M_2 &= M\beta^k \mod p
\end{align*}

On remarque qu'un message chiffré est deux fois plus long
que le message en clair. $k$ est une clé \emph{éphémère} !

Déchiffrement :
\begin{align*}
	M &= M_2 M_1^{-a} \mod p && \text{bof car division} \\
	&= M_2 M_1^{(p-1)-a} \mod p && \text{par Fermat (\ref{thm:fermat})}
\end{align*}

\begin{thm}[El Gamal]
	\[
		M_1^{-a}M_2 \equiv_p M
	\]
	\begin{proof}
		\begin{align*}
			M_1^{-a}M_2
			&\equiv_p \alpha^{-ak}M\beta^k && \text{définition de } M_1, M_2 \\
			&\equiv_p \alpha^{-ak}M\alpha^{ak} && \text{définition de } \beta \\
			&\equiv_p M && \alpha^{-ak}\alpha^{ak}=\alpha^0=1
		\end{align*}
	\end{proof}
\end{thm}

Selon \cite[p. 296]{hac}, si on intercepte deux messages chiffrés
$E(M)=M_1,M_2$ et $E(N)=N_1,N_2$, que $M$ est connu et $k$, réutilisé,
on peut retrouver $N$ :
\begin{align*}
	\frac{N_2}{M_2} &= \frac{N\beta^k}{M\beta^k} = \frac NM
		&& \text{définition} \\
	N &= M\frac{N_2}{M_2}
		&& \text{algèbre} \\
	N &\equiv_p MN_2M_2^{-1}
		&& \text{division dans }\esF_p^*
\end{align*}


\bibliographystyle{alpha}
\bibliography{refs}

\end{document}
